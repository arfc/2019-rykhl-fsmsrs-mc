\documentclass{anstrans}
%%%%%%%%%%%%%%%%%%%%%%%%%%%%%%%%%%%
\title{FUEL CYCLE PERFORMANCE OF FAST SPECTRUM \\
  MOLTEN SALT REACTOR DESIGNS}
\author{Andrei Rykhlevskii,$^{*}$ Benjamin R. Betzler,$^{\dagger}$ Andrew Worrall,$^{\dagger}$ and Kathryn Huff,$^{*}$}

\institute{
$^{*}$Dept. of Nuclear, Plasma, and Radiological Engineering, University of Illinois at 
  Urbana-Champaign, Urbana, IL \\ andreir2@illinois.edu
\and
$^{\dagger}$Oak Ridge National Laboratory, 1 Bethel Valley Road, Oak Ridge, TN, USA
}

%%%% packages and definitions (optional)
\usepackage{graphicx} % allows inclusion of graphics
\usepackage{booktabs} % nice rules (thick lines) for tables
\usepackage{microtype} % improves typography for PDF
\usepackage{cleveref}
\usepackage[acronym,toc]{glossaries}
\include{acros}

\makeglossaries
\newcommand{\SN}{S$_N$}
\renewcommand{\vec}[1]{\bm{#1}} %vector is bold italic
\newcommand{\vd}{\bm{\cdot}} % slightly bold vector dot
\newcommand{\grad}{\vec{\nabla}} % gradient
\newcommand{\ud}{\mathop{}\!\mathrm{d}} % upright derivative symbol

\begin{document}
%%%%%%%%%%%%%%%%%%%%%%%%%%%%%%%%%%%%%%%%%%%%%%%%%%%%%%%%%%%%%%%%%%%%%%%%%%%%%%%%
\section{INTRODUCTION} 
\label{sec:intro}
A liquid-fueled \gls{MSR} concepts promise one of the most desirable and competitive, sustainable energy among of many advanced reactor systems \cite{siemer_why_2015}. In \gls{MSR} fissile and/or fertile materials are dissolved in carrier molten salt (e.g., LiF, NaCl) which leads to immediate advantages over traditional, solid-fueled, reactors. These include near-atmospheric pressure in the primary loop, relatively high coolant temperature, outstanding neutron economy, a high level of inherent safety,
reduced fuel preprocessing, and the ability to continuously remove fission products and add fissile and/or fertile elements without shutdown \cite{leblanc_molten_2010}. 

Historically, researchers focused on thermal spectrum \gls{MSR} concepts with solid graphite moderator.  \gls{ORNL} operated $\approx$8 MW$_{th}$ \gls{MSRE} pilot reactor from 1965 to 1969 to test approaches/materials, demonstrate fissile recycle (both $^{233}$U and $^{235}$U), and determine generic \gls{MSR} operational characteristics \cite{macpherson_molten_1985}. Obtained experience plus promising breakthrough in reprocessing technology \cite{whatley_engineering_1970} chained \gls{ORNL} attention to the simply configured, also graphite-moderated (i.e., thermal or epithermal), single-fluid \gls{MSBR} by the end of the 1960's. The primary weakness of all one-fluid thermal \gls{MSR} concepts is the fact that the hundreds tons ($\approx$300 t for \gls{MSBR} \cite{robertson_conceptual_1971}) of expensive, radiologically contaminated, neutron-damaged graphite would have to be replaced every 4-10 years, which raises significant waste and economical issues.

In contrast, consistent with \gls{GIF} sustainability and safety goals \cite{gif_generation_2015}, unmoderated (no graphite), one- or two-fluid (blanket equipped) fast spectrum \gls{MSR} concept eventually became the EUROATOM Consortium's ``reference" \gls{MSR} \cite{noauthor_final_2015}. Other identified advantages of fast \gls{MSR} systems comparing with conventional reactors include: (1) they can operate in breeding (CR\footnote{\gls{CR} $\equiv$ fissile generated/fissile consumed: if CR$<$1 the reactor is a ``converter"; CR$\equiv$1, an ``isobreeder"; CR$>$1, a ``breeder".}$>$1) \cite{noauthor_final_2015,simmons_assessment_1974,mourogov_potentialities_2006}, fertile-free \gls{TRU} burning, or U/Th-supported \gls{TRU} burning \cite{ignatiev_progress_2007} regime; (2) they can potentially work in both $^{232}$Th/$^{233}$U and $^{238}$U/$^{239}$Pu fuel cycles (i.e., thermal \gls{MSR} supports only Th/U); (3) they can be operated in ways that would generate very little long-lived \gls{TRU} waste, and (4) they would use less natural resources (e.g., natural uranium, natural thorium) per unit energy generated. To quantitatively estimate benefits of these capabilities fuel cycle performance analysis for various fast spectrum \gls{MSR} concepts is necessary.

Much of the analysis herein uses unit cell representations of four different fast \gls{MSR} designs: 
(1) European \gls{MSFR} \cite{noauthor_final_2015}; (2) \gls{MCSFR} \cite{simmons_assessment_1974}; (3) REBUS-3700 \cite{mourogov_potentialities_2006}; (4) \gls{MOSART} \cite{ignatiev_progress_2007}. Some of these designs are two-fluid (1, 2), operate in thorium fuel cycle (1,4), or use \gls{TRU} as start-up fissile material. The concepts (2) and (3) use chloride salts\footnote{The chlorine in the \gls{MCSFR} is fully enriched in $^{37}$Cl because $^{35}$Cl (76\% abundance) is a very strong neutron poison in fast neutron energy range.} while (1) and (2) employ fluorides. This paper discusses the depletion simulation of different fast \glspl{MSR} to identify fuel cycle performance for deployment of this reactor technology.

\section{METHODS} 
\label{sec:methods}
To determine Fuel Cycle Performance parameters full-core and simplified unit cell models of four different fast spectrum \glspl{MSR} designs are created (Fig.~\ref{fig:unit_cell}).  Table~\ref{table:fsmsr_concepts} contains summary of principal data of these designs. Fuel cycle performance analysis required depletion simulation for each reactor concept over the system whole lifetime which in this work assumed 60 years\footnote{Lifetime for fast \gls{MSR} limited by neutron damage on the reactor vessel. Further R\&D activities should be performed to evaluate maximum fast neutron fluence and assess structural materials choice.}. Full-core 60-year depletion computation of \gls{MSR} required massive computing power. Computation time can be significantly reduced by performing depletion simulation for simplified unit cell representation instead of full-core one with sophisticated geometry.

All salt treatments and separation in this work are performed as truly continuous (online) processes using SCALE/TRITON version 6.2.4 Alpha with 238 group ENDF VII.1 cross-section library \cite{rearden_scale_2016}.  Earlier researchers usually used batch-wise approach because most of the reactor physics burnup calculation codes designed to model batch-fueled reactors with solid fuel \cite{betzler_molten_2017, rykhlevskii_online_2017}. This approach requires short depletion time steps to minimize the impact of no feed or removal during the time step which leads to a large number of depletion time steps to simulate longer periods of time. It makes lifetime-long depletion calculations computationally expensive. ORIGEN \cite{gauld_isotopic_2011} supports continuous feed and removal but does not track the feed and discharge materials that are important to fuel cycle performance analysis. Thus, alpha version of SCALE/TRITON currently developing in \gls{ORNL} allows simulate lifetime-long truly continuous depletion for \gls{MSR} with reasonable time steps and computation cost.
\begin{figure}[!htb]
  \centering
  \includegraphics[scale=0.265]{./Figures/fsmsrs.pdf}
  \caption{Full-core 3D models of \gls{MSFR} (upper left), \gls{MCSFR} (lower left), REBUS-3700 (upper right), \gls{MOSART} (lower right), and 2D unit cell model (center) showing fuel salt (red), fertile salt (green), structural material (blue).}   
  \label{fig:unit_cell}
\end{figure}
\begin{table*}[!htb]
  \centering
  \caption{Principal data of selected fast spectrum \glspl{MSR} designs.}
  \label{table:fsmsr_concepts} 
  \begin{tabular}{p{0.27\textwidth} p{0.14\textwidth} p{0.18\textwidth} p{0.14\textwidth} p{0.15\textwidth}} \toprule 
   Parameter & \gls{MSFR} & \gls{MCSFR} & REBUS-3700 & \gls{MOSART} \\ \midrule
   Thermal power, MW 				&  3,000 & 6,000     & 3,700 & 2,400   \\ 
   Fuel salt volume (in/out), m$^3$       &18 (9/9)& 38 (16/22)& 55.6 (36.9/18.7) & 49.05 (32.7/16.35) \\ 
   Fertile salt volume (in/out), m$^3$ & 7.3 (7.3/0) & 75 (55/22)    & --- & --- \\
   Fuel and fertile salt initial composition (mol\%) & LiF-ThF$_4$-$^{233}$UF$_4$ (77.5-19.9-2.6) LiF-ThF$_4$ \newline (77.5-22.5) & NaCl-UCl$_3$-$^{239}$PuCl$_3$ (60-36-4) \newline NaCl-UCl$_3$ \newline (60-40)    
   & 55mol\%NaCl+ 45mol\%(natU+ 16.7at.\%TRU)Cl$_3$ 
   & LiF-BeF$_2$-ThF$_4$-TRUF$_3$  \newline (69.72-27-1.28) \\
   Fuel cycle & Th/$^{233}$U & U/Pu  & U/TRU & Th/$^{233}$U \\
   Initial fissile inventory, t & \ 5.060 & \ 9.400    & 18.061 & 9.637 \\ \bottomrule 
  \end{tabular}
\end{table*}

\subsection{Models description} 
\label{sec:model}
In contrast with thermal \gls{MSR}, fast spectrum concepts do not have channel or assembly structure but contain homogenized fuel mixture into cylindrical or spherical vessel. Two-fluid systems also have cylindrical (\gls{MSFR}) or spherical (\gls{MCSFR}) blanket with fertile salt to reduce neutron leakage and enhance fissile material breeding (Fig.~\ref{fig:unit_cell}). Details about reactors' configuration can be found in Refs.~\cite{noauthor_final_2015, simmons_assessment_1974, mourogov_potentialities_2006,ignatiev_molten_2014}. Two-fluid concepts' 2D unit cell model contains a cylindrical fuel salt channel with thin outer layer of fertile salt inside square block of structural material (Hastelloy N). The unit cell model for single-fluid REBUS-3700 has fuel salt and structural material only; \gls{MOSART} simplified model consist of fuel salt square block with graphite cylinder in the center to represent 0.2 m graphite reflector which needed to increase $^{233}$U breeding from thorium. 

To prove viability of unit cell models for depletion simulation high-fidelity full-core models were developed using Monte Carlo code SERPENT2 (16 millions neutron histories per run) with ENDF/B-VII.1 library \cite{leppanen_serpent_2015, chadwick_endf/b-vii.1_2011}. Single average unit cell model geometry and size are optimized to obtain sufficiently accurate multiplication factor and neutron energy spectrum in a reasonable time. Next metrics are used for optimization:
\vspace{-0.06in}
\begin{enumerate}
	\item eigenvalue discrepancy between full-core and unit cell models less than 300 pcm\footnote{ 1 pcm = 10$^{-5}\Delta k_{eff}/k_{eff}$};\vspace{-0.07in}
	\item correlation coefficient (r) for neutron spectrum normalized by lethargy more than 0.995;\vspace{-0.07in}
	\item approximation error ($\delta$) in total neutron flux less than 3\%.\vspace{-0.06in}
\end{enumerate}
The symmetry in a reactors is used to simplify the problem into one-quarter of the unit cell geometry. For this optimization the 16-by-16 spatial mesh for the NEWT neutron transport calculation is used in SCALE/TRITON.

\subsection{Fuel Cycle Performance metrics} 
\label{sec:metrics}
The main objective of the work is to analyze fast \gls{MSR} systems and fuel cycles in support of the Fuel Cycle Options Campaign of the US Department of Energy, Office of Nuclear Energy (DOE-NE). The Evaluation and Screening Study (E\&S) conducted by the DOE-NE gives information about the potential benefits and challenges of nuclear fuel cycle options (i.e., closed fuel cycle with continuous \gls{MA} reprocessing). This information needed to strengthen the basis and provide guidance for the activities started by the DOE-NE Fuel Cycle Research and Development program.

DOE established an Evaluation and Screening Team (EST) consisting of national laboratory and industry experts in nuclear fuel cycles to develop the Evaluation Metrics for 40 various nuclear fuel cycle options (``Evaluation Groups"). Based on continuous reprocessing depletion simulations, conducted for 4 selected fast \gls{MSR} designs, following Evaluation Metrics are determined:
\vspace{-0.06in}
\begin{enumerate}
	\item Natural Uranium per energy generated (for \gls{MCSFR}, REBUS-3700);\vspace{-0.07in}
	\item Natural Thorium per energy generated (for \gls{MSFR}, \gls{MOSART});\vspace{-0.07in}
	\item Mass of SNF+HLW\footnote{\gls{SNF}+\gls{HLW}} disposed per energy generated;\vspace{-0.06in}
	\item Mass of DU+Ru+RTh\footnote{\gls{DU}+\gls{RU}+\gls{RTh}} disposed per energy  generated;\vspace{-0.05in}
\end{enumerate}
Obtained results are compared with Metric Data for respective Evaluation Groups from Nuclear Fuel Cycle Evaluation and Screening study \cite{wigeland_nuclear_2014}.

\section{RESULTS} 
This section presents calculation results, such as neutron flux spectrum for full-core and unit cell models, fuel and fertile salt composition evolution during the reactor operation, and evaluation metrics.

\subsection{Full-core vs unit cell neutron spectrum} 
\label{sec:spectrum}
\Cref{fig:spectrum_msfr,fig:spectrum_mcsfr,fig:spectrum_rebus,fig:spectrum_mosart} show neutron flux energy spectrum normalized by lethargy for full-core 3D models obtained with SERPENT2 and simplified unit cell 2D models obtained with SCALE/TRITON. Calculated correlation coefficient ($r$) and approximation error ($\delta$) in total neutron spectrum for unit cell model indicated in right upper corner of each plot. The most accurate approximation is obtained for the \gls{MOSART}, while the \gls{MCSFR} has the worse approximation quality. The main reason of discrepancy is low energy groups resolution in fast energy region in 238 group nuclear data library which was developed for thermal reactors. This issue could be resolved by using problem-oriented multi-group cross-section libraries for SCALE/TRITON for fast reactors but it would significantly increase computational time.
\begin{figure}[!htb]
  \centering
  \includegraphics[scale=0.585]{./Figures/msfr_full_vs_unit_spectrum.png}
  \caption{Neutron flux energy spectrum for full-core and unit cell models for two-fluid \gls{MSFR}.}   
  \label{fig:spectrum_msfr}
\end{figure}
\begin{figure}[!htb]
  \centering
  \includegraphics[scale=0.585]{./Figures/mcsfr_full_vs_unit_spectrum.png}
  \caption{Neutron flux energy spectrum for full-core and unit cell models for two-fluid \gls{MCSFR}.}   
  \label{fig:spectrum_mcsfr}
\end{figure}
\begin{figure}[!htb]
  \centering
  \includegraphics[scale=0.58]{./Figures/rebus_full_vs_unit_spectrum.png}
  \caption{Neutron flux energy spectrum for full-core and unit cell models for single-fluid REBUS-3700.}   
  \label{fig:spectrum_rebus}
\end{figure}
\begin{figure}[!htb]
  \centering
  \includegraphics[scale=0.585]{./Figures/mosart_full_vs_unit_spectrum.png}
  \caption{Neutron flux energy spectrum for full-core and unit cell models for single-fluid \gls{MOSART} with graphite reflector.}   
  \label{fig:spectrum_mosart}
\end{figure}

\subsection{Fuel Cycle Performance analysis} 
\label{sec:performance}
\Cref{fig:k_inf} 

\begin{figure}[!htb]
  \centering
  \includegraphics[scale=0.585]{./Figures/k_inf.png}
  \caption{Infinite multiplication factor converging to equilibrium after 35 years of
operation with full separations and feeds.}   
  \label{fig:k_inf}
\end{figure}
\begin{table*}[!htb]
  \centering
  \caption{Fuel Cycle Performance metrics of selected fast spectrum \glspl{MSR} designs.}
  \label{table:metrics} 
  \begin{tabular}{p{0.4\textwidth} p{0.13\textwidth} p{0.13\textwidth} p{0.13\textwidth} p{0.13\textwidth}} \toprule 
   Parameter & \gls{MSFR} & \gls{MCSFR} & REBUS & \gls{MOSART} \\ \midrule
   Evaluation Group	&  EG28 & EG23 & EG24 & EG28   \\ 
   Natural Uranium or Thorium Utilization, t/GWe-yr & TBD & TBD & 0.835 (U) & 0.672 (Th) \\
   Mass of \gls{SNF}+\gls{HLW} disposed, t/GWe-yr & TBD & TBD & 0.813 &  0.821 \\
   Mass of \gls{DU}+\gls{RU}+\gls{RTh} disposed, t/GWe-yr & TBD & TBD & 0.0 & 0.0 \\
   Products from Reprocessing/Separation technology, t: \gls{RU}/\gls{RTh}/\gls{TRU}/\gls{FP} &
   \ \newline TBD & \ \newline TBD & \ \newline 92.3/0/19.2/79.6 & \ \newline 4.2/17.7/6.3/52.1  \\ \bottomrule 
  \end{tabular}
\end{table*}


Equations, such as Eq. (\ref{sample_equation}), should be centered and 
sequentially numbered to the flush right of the formula.

\begin{equation}
  \label{sample_equation}
  \mathrm{Speedup}=\frac{1}{\frac{f}{p}+(1-f)}
\end{equation}

The continuation of a paragraph after an equation should not be indented.  
All paragraphs, as well as section or subsection headings, are separated by 
just one single empty line.

\subsubsection{Sub-subsection level and lower: only first character uppercase}

See Table \ref{table:example} for a sample table.  The ``tabls'' package is
recommended for improved row and column spacing.  Notice the caption appears 
above the table by setting the \verb!\caption! command immediately 
after the \verb!\begin{table}!. Tables are numbered in Roman 
numerals, with the caption centered above the table, in \textbf{boldface}.  
Triple-space before and after the table.

\section{CONCLUSIONS}

Present your summary and conclusions here.
%%%%%%%%%%%%%%%%%%%%%%%%%%%%%%%%%%%%%%%%%%%%%%%%%%%%%%%%%%%%%%%%%%%%%%%%%%%%%%%%
\section{Acknowledgments}
This research was supported by the Fuel Cycles Options Campaign of the Fuel Cycle Technologies initiative 
and by an appointment to the Oak Ridge National Laboratory Nuclear Engineering Science Laboratory Synthesis (NESLS) Program, sponsored by U.S. Department of Energy and administered by the Oak Ridge Institute for Science and Education.

%%%%%%%%%%%%%%%%%%%%%%%%%%%%%%%%%%%%%%%%%%%%%%%%%%%%%%%%%%%%%%%%%%%%%%%%%%%%%%%%
\bibliographystyle{ans}
\bibliography{2019-rykh-fsmsrs-mc}
\end{document}

